% Ensemble et op�rateur avec un barre pr�c�dent le symbole
\newcommand{\E}{\mathbb{E}}     
\renewcommand{\P}{\mathbb{P}}     
\newcommand{\N}{\mathbb{N}}
\newcommand{\Q}{\mathbb{Q}}     
\newcommand{\Z}{\mathbb{Z}}
\newcommand{\R}{\mathbb{R}} 
\newcommand{\V}{\mathbb{V}} 
\newcommand{\1}{\mathds{1}}%n�cc�cite le package dsfont

% Lettres calligaphiques
\newcommand{\cA}{\mathcal{A}}
\newcommand{\cU}{\mathcal{U}}
\newcommand{\cN}{\mathcal{N}}
\newcommand{\cE}{\mathcal{E}}
\newcommand{\cC}{\mathcal{C}}
\newcommand{\cI}{\mathcal{I}}
\newcommand{\cO}{\mathcal{O}}

% Utilisation du diff�rentiel d
\renewcommand{\d}[1]{\,\ensuremath{\operatorname{d}\!{#1}}}

% D?finition des envrionement de type ?nonc?s num?rot?s simplement
\newtheorem{theo}{Th?or?me}
\newtheorem{rema}{Remarque}
\newtheorem{exem}{Exemple}
\newtheorem{defi}{D?finition}
\newtheorem{exer}{Exercice}
\newtheorem{prop}{Proposition}
\newtheorem{code}{Code}

% D?finition des m?mes envrionements de type ?nonc?s num?rot?s mais avec l'ajout pr?alable de la section 
\newtheorem{adef}{D?finition}[section]
\newtheorem{apro}{Proposition}[section]
\newtheorem{athe}{Th?or?me}[section]
\newtheorem{aexe}{Exercice}[section]
\newtheorem{aexem}{Exemple}[section]
\newtheorem{arem}{Commentaire}[section]

% D?finition d'un environement qui utilise le m?me compteur que l'environnement athe
\newtheorem{thm}[athe]{M?me compteur que Th?or?me }

% D?finition d'une commande qui met une boite blanche en but de ligne
\newcommand{\fdem}{\hspace*{\fill}~$\Box$\par\endtrivlist\unskip}

% D?finition d'un nouvelle environnement avec un argument. La fin de l'environnement est marqu? par le symbole g?n?r? par la commande \fdem 
\newenvironment{proof}[1]{\textit{Preuve#1.\,}}{\fdem}

% R�f�rencement fain�ant des figures
\newcommand{\tref}[1]{\textsc{Fig.} \ref{#1}}

%indiquer la source d'une figure avec \source{La source}
\newcommand{\source}[1]{\vspace{-10pt} \caption*{\footnotesize \textbf{Source:} {#1}} }

