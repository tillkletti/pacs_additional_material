\usepackage{lmodern}
\usepackage[latin1]{inputenc}
\usepackage[T1]{fontenc}
\usepackage[english]{babel}


%%%%%%%%%%%%%%%%%%%% T E X T E %%%%%%%%%%%%%%%%%%%%%%

% package qui permettent d'utiliser tout un tas de symboles math?matiques
\usepackage{amsmath}
\numberwithin{equation}{section}

\usepackage{amssymb}
\usepackage{empheq}
%\usepackage{mathrsfs}
\usepackage[squaren,Gray]{SIunits} %Unit�s SI

% Polices
%\usepackage{mathpazo} %Police styl�e de J�r�my Omer
%\usepackage{txfonts} %autre police styl�e, s'approche du style de word
%\usepackage{libertine}

% permet d'?crire des caract?res chinois
%\usepackage{CJKutf8}
%Utiliser le pinyin
%\usepackage{pinyin}

% package qui permettent de personaliser les puces des listes
\usepackage{enumerate}
\usepackage{enumitem} % pour utiliser description

% package qui permettent d'introduire de la couleur
\usepackage{color}
\definecolor{mygreen}{rgb}{0,0.6,0}
\definecolor{mygray}{rgb}{0.5,0.5,0.5}
\definecolor{mymauve}{rgb}{0.58,0,0.82}

\usepackage{xspace}
% permet l'affichage de l'heure courante
%\usepackage{datetime}

%pour pouvoir barrer avec \cancel{}
%\usepackage{cancel}

%pour la fonction indicatrice 1
%\usepackage{dsfont}

%pour utiliser le symbole euro par \euro
\usepackage{eurosym}

%afficher des nombres en mode texte
\usepackage{numprint}

%Capitale grande en d�but de paraghraphe
%\usepackage{lettrine}


%%%%%%%%%%%%%%%%%%%% F I G U R E S %%%%%%%%%%%%%%%%%%%%%%

% package qui permettent d'inclure des graphiques
\usepackage{graphicx}
\numberwithin{figure}{section}

%Pour utiliser des sous-figures
\usepackage{subcaption}

%Pour utiliser [H] pour le placement des figures
\usepackage{float}

%figure entour�e de texte
\usepackage{wrapfig}
% \begin{wrapfigure}{L}{0.4\textwidth}

%Keep figures in section
\usepackage[section]{placeins}
% \FloatBarrier % stops floats from descending further

% Pour des r�f�rences styl�es
%\usepackage{varioref}
%\usepackage[german]{fancyref} % dispo seulement en allemand
\usepackage[linkcolor=black,
						urlcolor=black,
						colorlinks=true,
						citecolor=mygreen]{hyperref}% package qui cr�er des liens dynamiques dans un fichier pdf � partir de tout �l�ment tagg� 
%\usepackage{cleveref}


%%%%%%%%%%%%%%%%%%%% D E S S I N %%%%%%%%%%%%%%%%%%%%%%

%pour dessiner des graphes de fonctions
\usepackage{pgfplots}

%pour plein de commandes de dession
\usepackage{tikz}
\usetikzlibrary{decorations.pathreplacing,calligraphy}

%%%%%%%%%%%%%%%%%%%% H E A D E R et F O O T E R %%%%%%%%%%%%%%%%%%%%%%

\usepackage{fancyhdr}
%\fancyfoot{}
\fancyfoot[LE,RO]{\thepage}


%%%%%%%%%%%%%%%%%%%%%% D I V E R S %%%%%%%%%%%%%%%%%%%%%%%%%%

%Bibliographie
\usepackage[backend=bibtex,sorting=ynt]{biblatex}%sorting=none pour avoir les r�f�rences selon d'ordre d'apparition dans le texte
% sorting=ynt		backend=biber,

%pour que les textes en Verbatim soient encadr�s
%\usepackage{fancyvrb}

% Mettre en forme son texte en plusieurs colonnes
%\usepackage{multicol}

%pour manipuler les compteurs avec par exemple \counterwithin{figure}{section}
\usepackage{chngcntr}

%pour importer des sections
\usepackage{import}

%Appendices
\usepackage[title,toc,page,header]{appendix}
\renewcommand{\appendixtocname}{Tables des Annexes}
\renewcommand{\appendixpagename}{Annexes}

%\usepackage{couvresumesPFE} % utilisation du package pfe.sty pour cr�er la couverture et la page de r�sum�s en fran�ais et en anglais


%%%%%%%%%%%%%%%%%%% C O D E %%%%%%%%%%%%%%%%%%%%%%%

% Pour �crire du pseudo-code
%\usepackage{algorithm2e}
%\usepackage{algorithmic}
\usepackage{algorithm}
\usepackage{algpseudocode}
%\listofalgorithms % commande pour lister les algorithmes

%pour ins�rer du code
\usepackage{listings}
\usepackage{listingsutf8}
%\usepackage{inconsolata}
\lstdefinestyle{CodeR}{ %
  backgroundcolor=\color{white},
  basicstyle=\footnotesize,
  breakatwhitespace=false
  breaklines=true,
  captionpos=b,
  commentstyle=\color{mygreen},
  deletekeywords={...},
  escapeinside={Ratus}{Chiffon},
  extendedchars=true,
  frame=single,
  keepspaces=true,
  keywordstyle=\color{blue},
  language=R,
  otherkeywords={*,...},
  numbers=left,
  numbersep=5pt,
  numberstyle=\tiny\color{mygray},
  rulecolor=\color{black},
  showspaces=false,
  showstringspaces=false,
  showtabs=false,
  stepnumber=1,
  stringstyle=\color{mymauve},
  tabsize=4,
  title=\lstname
}
\lstdefinestyle{CodeMATLAB}{ %
  backgroundcolor=\color{white},
  basicstyle=\footnotesize,
  breakatwhitespace=false
  breaklines=true,
  captionpos=b,
  commentstyle=\color{mygreen},
  deletekeywords={...},
  escapeinside={James}{Ledoux},
  extendedchars=true,
  frame=single,
  keepspaces=true,
  keywordstyle=\color{blue},
  language=MATLAB,
  otherkeywords={*,...},
  numbers=left,
  numbersep=5pt,
  numberstyle=\tiny\color{mygray},
  rulecolor=\color{black},
  showspaces=false,
  showstringspaces=false,
  showtabs=false,
  stepnumber=1,
  stringstyle=\color{mymauve},
  tabsize=4,
  title=%\lstname
}
\lstset{style=CodeMATLAB}

